%---------- Inleiding ---------------------------------------------------------

% --- Is dit voorstel gebaseerd op een paper van Research Methods die je
% vorig jaar hebt ingediend? Heb je daarbij eventueel samengewerkt met een
% andere student?
% Zo ja, haal dan de tekst hieronder uit commentaar en pas aan.

\paragraph{Opmerking}

Dit voorstel is gebaseerd op het onderzoeksvoorstel dat werd geschreven in het
kader van het vak Research Methods dat ik dit academiejaar heb
uitgewerkt (met medestudent Sam De Gendt als mede-auteur).

\section{Inleiding}
\label{sec:inleiding}

% Context schetsen
Trampolinespringen is een technische en complexe sport, waar atleten beoordeeld worden op basis van verschillende componenten zoals moeilijkheidsgraad, uitvoering, horizontale verplaatsing (horizontal displacement, HD) en vluchttijd (time of flight, ToF) \autocite{FTTC2024}. 

Het meten van HD en ToF in officiële competities gebeurt door middel van het \emph{HDTS Trampoline Measurement Device}. Dit systeem, verder afgekort als HDTS, gebruikt vier krachtplaten die geïnstalleerd zijn onder de poten van de trampoline. Het systeem berekent zowel de HD als ToF aan de hand van krachtverhoudingen en timing gemeten door de vier platen \autocite{Ferger2019}.

% Probleemstelling
Het probleem met het huidige HDTS-systeem is de kostprijs, die ligt tussen de \euro{}5.000 en \euro{}10.000\footnote{\url{https://gymaid.com/products/eurotramp-hdts/}}. Dit maakt het systeem voor veel turnclubs onbetaalbaar. Uit een recent artikel blijkt dat de sportsector onder druk staat: de Vlaamse Sportfederatie waarschuwt voor de financiële impact van besparingen op sportclubs, wat de investeringsruimte voor dergelijke apparatuur verder beperkt \autocite{Sporza2025}.

Turnclubs met beperkte budgetten kunnen zich hierdoor geen HDTS-systeem veroorloven. Dit creëert een ongelijkheid, aangezien atleten in deze clubs geen toegang hebben tot automatische metingen van HD en ToF tijdens hun trainingen. Zij zijn genoodzaakt om hun prestaties te analyseren via tijdsintensieve, manuele video-analyses, wat inefficiënt is.

% Hoofdvraag
Dit onderzoek richt zich daarom op de volgende hoofdvraag: 
\textit{Hoe kan een goedkoop computer vision-systeem worden ontwikkeld dat horizontale verplaatsing en vluchttijd bij trampolinespringen accuraat kan meten, als betaalbaar alternatief voor het HDTS-meetsysteem?}

% Deelvragen
Om een antwoord te vinden op deze hoofdvraag, wordt het onderzoek opgesplitst in twee domeinen: het probleemdomein en het oplossingsdomein.

\subsection*{Deelvragen: Probleemdomein}
Allereerst is het noodzakelijk om de context en de vereisten van het probleem volledig te begrijpen. De volgende deelvragen richten zich op het in kaart brengen van de huidige situatie en de randvoorwaarden voor een alternatief systeem \autocite{Ferger2022}:
\begin{enumerate}
    \item Hoe wordt de scorecomponent voor horizontale verplaatsing exact berekend in competitieverband?
    \item Hoeveel variatie is er in trampoline-afmetingen en opstellingen waarmee het systeem rekening moet houden?
    \item Welk aandeel van Vlaamse clubs beschikt niet over een HDTS en wat zijn de daadwerkelijke kosten (aankoop en onderhoud)?
    \item Welke nauwkeurigheid moet een alternatief systeem minimaal behalen om van toegevoegde waarde te zijn voor trainingen?
    \item Is een real-time verwerking van de data vereist tijdens de training?
\end{enumerate}

\subsection*{Deelvragen: Oplossingsdomein}
Vervolgens wordt gekeken naar de technische realisatie van een alternatief systeem met behulp van computer vision. Hierbij staan de volgende technische vragen centraal:
\begin{enumerate}
    \item Welke hardware-configuratie (cameraresolutie, framerate en positie) levert de hoogste nauwkeurigheid op?
    \item Welke datasetgrootte en welk labelformat zijn vereist om een betrouwbaar basismodel te trainen?
    \item Welke modelarchitectuur en pre-processing technieken zijn het meest geschikt om de landingspositie en scores te voorspellen?
    \item Is het effectiever om contactmoment- en positiedetectie te scheiden, of biedt een end-to-end model betere resultaten?
    \item Welke data-augmentaties kunnen de robuustheid van het model verbeteren?
\end{enumerate}

\subsection*{Onderzoeksdoelstelling}
Het doel van dit onderzoek is om een proof of concept te ontwikkelen van een kosteneffectief alternatief voor commerciële HDTS-systemen. Hierbij wordt gestreefd naar een systeem dat de horizontale verplaatsing en vluchttijd kan bepalen met een betrouwbaarheid van meer dan 80\%, binnen het budget van een gemiddelde sportclub.


\section{Literatuurstudie}
\label{sec:literatuurstudie}

Deze korte literatuurstudie is opgedeeld in twee delen. In het eerste deel wordt de scoreberekening en de reglementering binnen het trampolinespringen toegelicht. Het tweede deel omvat de huidige stand van zaken (state-of-the-art) op het gebied van meettechnologieën.

\subsection{Scoreberekening en Reglementering}
\label{subsec:scoreberekening}

Om het belang van HD en ToF bij de beoordeling te begrijpen, wordt eerst het algemene beoordelingsproces toegelicht zoals vastgelegd door de internationale gymnastiekfederatie.

\subsubsection{Algemene Beoordeling}
Volgens de FIG Code of Points \parencite{FTTC2024} bestaat een individuele oefening uit een reeks van 10 sprongen, waarvan de eindscore wordt bepaald op basis van vijf componenten:

\begin{itemize}
    \item $D$ - Moeilijkheidsgraad (Difficulty)
    \item $E$ - Uitvoering (Execution)
    \item $HD$ - Horizontale verplaatsing (Horizontal Displacement)
    \item $ToF$ - Vluchttijd (Time of Flight)
    \item $P$ - Strafpunten (Penalties)
\end{itemize}

De berekening van de eindscore gebeurt aan de hand van de volgende formule:

\[
\text{Eindscore} = E + HD + D + ToF - P
\]

In dit onderzoek ligt de focus specifiek op de HD en ToF componenten.

\subsubsection{Horizontale Verplaatsing (HD)}
De HD component geeft weer hoe vaak een gymnast tijdens zijn of haar sprongreeks afweek van het midden van de trampoline. Het trampolinebed is opgedeeld in zones. Afhankelijk van de zone waarin de voeten van de atleet landen, worden er strafpunten afgetrokken van een basisscore van 10 punten \autocite{FTTC2024}.

Het midden van de trampoline geldt als ideale landingspositie zonder aftrek, terwijl elke zone die verder ligt van het midden geleidelijk hogere strafpunten oplevert. Figuur~\ref{fig:HD_zones} illustreert de hoeveelheid punten die per zone worden afgetrokken van de basisscore.

\begin{figure}[ht]
    \centering
    \includegraphics[width=0.7\linewidth]{"./img/HD_zones.jpg"}
    \caption{Indeling van zones op het trampolinebed voor het bepalen van strafpunten voor horizontale verplaatsing \autocite{FTTC2024}.}
    \label{fig:HD_zones}
\end{figure}

\subsubsection{Vluchttijd (ToF)}
De ToF component omvat de totale tijd dat de gymnast zich in de lucht bevindt tijdens zijn of haar routine. Deze is belangrijk om twee redenen: enerzijds draagt ToF direct bij aan de totaalscore, anderzijds betekent een hogere ToF dat de gymnast meer tijd heeft om de technische componenten uit te voeren.

Deze tijd wordt geregistreerd in duizendsten van een seconde en vervolgens afgerond naar beneden op een honderdste van een seconde. Zo zou een gymnast met een gemeten vluchttijd van 16.197 seconden een score krijgen van 16.19 voor ToF \autocite{FTTC2024}:
\[
16.197 \text{ seconden} \rightarrow \lfloor 16.197 \rfloor_{\downarrow 0.01} = 16.19 \text{ punten}
\]

\subsection{State-of-the-art Technologieën}
\label{subsec:state_of_the_art}

Er zijn verschillende methoden om de hierboven beschreven parameters te meten. Hieronder worden het huidige commerciële standaardsysteem en recent relevant onderzoek naar alternatieve camera-gebaseerde methoden besproken.

\subsubsection{Huidige Standaard: HDTS}
\textcite{Ferger2017} beschrijven hoe de introductie van objectieve meetwaarden voor HD en ToF heeft geleid tot de ontwikkeling van het \emph{HDTS Trampoline Measurement Device}. Dit systeem, gekoppeld aan de Eurotramp \textit{Qira 1.03} software, geldt momenteel als de standaard in competities.

Het systeem maakt gebruik van vier krachtplaten die geplaatst worden onder de poten van het trampolinerek. Voorafgaand aan gebruik kalibreert het systeem zichzelf automatisch. Op basis van de timing en verdeling van de gemeten krachten via de sensoren worden de waarden voor zowel HD als ToF berekend en real-time weergegeven \autocite{Dyas2023}.

\paragraph{Berekening Horizontale Verplaatsing}
De HD-waarde wordt bepaald door de landingsposities te berekenen op basis van de krachtverhoudingen. De coördinaten worden berekend met de volgende formules \autocite{Ferger2019}:

\begin{align*}
    F(dx) &= \frac{F_3 + F_4}{\frac{L}{2}} - \frac{F_1 + F_2}{\frac{L}{2}} \\ 
    F(dy) &= \frac{F_2 + F_3}{\frac{B}{2}} - \frac{F_1 + F_4}{\frac{B}{2}}
\end{align*}

Waarbij $F_{1\dots4}$ de krachten van de vier platen zijn, en $L$ en $B$ respectievelijk de lengte en breedte van de trampoline. De oriëntatie van deze platen is weergegeven in Figuur~\ref{fig:Krachtplaten}.

\begin{figure}[ht]
    \centering
    \includegraphics[width=0.7\linewidth]{"./img/Krachtplaten.jpg"}
    \caption{Positie van de krachtplaten onder de trampoline en de X- en Y-richtingen voor positiebepaling \autocite{Ferger2019}.}
    \label{fig:Krachtplaten}
\end{figure}

\paragraph{Berekening Vluchttijd}
ToF wordt gemeten door de krachtmetingen te combineren met een algoritme dat landings- en afstootmomenten detecteert. Het systeem zoekt naar een krachtpiek (>5000 N) en bepaalt de contactmomenten op het punt waar 50\% van die piek wordt bereikt. Vanwege de elasticiteit van het doek wordt de landing geschat op 74 ms vóór dit punt en de afstoot op 88 ms erna (zie Figuur~\ref{fig:krachtcurve}) \autocite{Ferger2019}.

\begin{figure}[ht]
    \centering
    \includegraphics[width=0.5\textwidth]{./img/ToF_meting.jpg}
    \caption{Kracht-tijd curve ter illustratie van de schatting van afstoot en landing \autocite{Ferger2019}.}
    \label{fig:krachtcurve}
\end{figure}

\subsubsection{Computer Vision Benaderingen}
Het onderzoek van \textcite{Park2022} presenteert een systeem dat driedimensionale voetposities op een fitnesstrampoline bepaalt. Dit is relevant om de haalbaarheid van een camera-gebaseerd alternatief voor HDTS te evalueren.

Dit systeem gebruikt een camera met fisheye-lens onder de trampoline om schaduwen van de voeten te registreren. De beelden worden omgezet naar grijswaarden en genormaliseerd. Vervolgens voorspelt een deep learning-model de coördinaten. 

Figuur~\ref{fig:park_pipeline} toont de werking van dit algoritme: naast de visuele data wordt ook de voetgrootte als parameter meegenomen om de nauwkeurigheid van het Faster R-CNN netwerk te verhogen.

\begin{figure}[ht]
    \centering
    \includegraphics[width=0.6\linewidth]{"./img/ParkAlgoritme.jpg"}
    \caption{Schematische weergave van het detectie-algoritme. De fisheye-beelden worden omgezet naar grijswaarden en samen met de voetgrootte verwerkt door een Faster R-CNN netwerk om de XYZ-coördinaten te bepalen \autocite{Park2022}.}
    \label{fig:park_pipeline}
\end{figure}

De resultaten van dit onderzoek tonen een gemiddelde afwijking van 11.7 mm. Dit suggereert dat deze methode, mits aangepast voor grote trampolines, voldoende nauwkeurig kan zijn voor het bepalen van de landingszones.

% Refereren naar de literatuur kan met:
% \autocite{BIBTEXKEY} => (Auteur, jaartal): voor een referentie tussen
% haakjes, waar de auteursnaam GEEN onderdeel is van een zin.
% \textcite{BIBTEXKEY} => Auteur (jaartal): voor een narratieve referentie,
% waar de naam van de auteur effectief een onderdeel is van de zin.

\section{Methodologie}
\label{sec:methodologie}

% TODO: Methodologie feedback

Om de risico's te beperken en flexibel in te spelen op nieuwe inzichten, wordt een Agile-werkwijze gehanteerd. In tegenstelling tot een watervalmodel, lopen literatuurstudie, dataverzameling en ontwikkeling gedeeltelijk parallel en beïnvloeden ze elkaar continu. Het onderzoek is opgedeeld in een voorbereidende fase en een iteratieve uitvoeringsfase.

Een schematisch overzicht van deze planning is weergegeven in Figuur~\ref{fig:fasenverloop}.

\begin{figure*}
    \centering
    \includegraphics[width=\textwidth]{./img/gantt-PVA.png}
    \caption{\label{fig:fasenverloop}Gantt-diagram met de iteratieve fasen en mijlpalen van het onderzoek.}
\end{figure*}

\subsection{Fase 1: Probleemanalyse \& Validatie}
Deze fase focust op het beantwoorden van de deelvragen binnen het probleemdomein en de initiële validatie van de technische haalbaarheid.

\subsubsection{Requirements Analyse}
Om de requirements op te stellen, wordt onderzocht aan welke minimale nauwkeurigheidseisen een alternatief systeem moet voldoen, wat het beschikbare budget bedraagt, en wat de effectieve noodzaak is binnen trampolineclubs.

\subsubsection{Technische Specificaties}
Hierbij wordt gefocust op de exacte berekeningswijze van horizontale verplaatsing volgens officiële reglementen en de variatie in trampoline-opstellingen. Daarnaast worden de financiële aspecten, zoals de aankoop- en onderhoudskosten van het huidige HDTS-systeem, in kaart gebracht.

\subsubsection{Hardware \& Eisen}
Op basis van bovenstaand onderzoek zijn de functionele eisen en voorwaarden al deels vastgelegd. Daarnaast werden mogelijke camera-opstellingen reeds besproken met trainers om zowel de kwaliteit van de data als de veiligheid van gymnasten te waarborgen.

\subsubsection{Initiële Dataset (PoC)}
Om de haalbaarheid van het oplossingsdomein vroegtijdig te toetsen, werd gedurende het eerste semester reeds de eerste data verzameld. Tijdens deze tests werd een beperking in de \textit{Field of View} (FOV) vastgesteld. Om dit probleem te verhelpen is een fisheye-lens aangeschaft. Met deze aangepaste configuratie kon nieuwe, bruikbare data worden opgenomen die als basis dient voor de eerste iteraties.

\subsection{Fase 2: Iteratieve Ontwikkeling}
In deze fase ligt de focus op de deelvragen van het oplossingsdomein (modelarchitectuur, pre-processing, data-augmentatie). Hierbij wordt er experimenteel gezocht naar de optimale dataverwerking en dataflow door middel van iteratieve sprints. Elke sprint levert inzichten op die direct worden meegenomen in de volgende iteratie.

\subsubsection{Onderzoek Architectuur \& Strategie}
Voor de start van de iteraties wordt via literatuuronderzoek bepaald welke modelarchitectuur en welk labelformaat het meest geschikt zijn. Dit onderzoek vormt de basis voor de eerste iteratie, maar blijft een doorlopende activiteit om latere iteraties te ondersteunen.

\subsubsection{Iteratie 1: Baseline MVP}
De dataset uit de PoC-fase wordt gelabeld volgens de gekozen strategie. Een eerste prototype (MVP) wordt getraind om de effectiviteit van de gekozen architectuur te valideren.

\subsubsection{Iteratieve Cyclus: Experimentele Validatie}
Om de nauwkeurigheid stapsgewijs te maximaliseren, wordt gewerkt in opeenvolgende iteraties. Elke iteratie start met een analyse van het vorige model, gevolgd door gerichte experimenten:
\begin{itemize}
    \item \textbf{Data \& Pre-processing:} Uitbreiden van de dataset voor ondervertegenwoordigde klassen en testen van augmentatietechnieken (o.a. rotatie, belichting) voor betere robuustheid.
    \item \textbf{Modelarchitectuur:} Aanpassen van de architectuur op basis van de geïdentificeerde knelpunten, vergeleken t.o.v. de baseline.
\end{itemize}
Dit cyclische proces beantwoordt de deelvragen over welke configuratie de hoogste nauwkeurigheid oplevert.

\subsection{Finale Validatie}
\label{sec:validatie}
Het finale model wordt gevalideerd op een aparte testset die doorheen het proces is samengesteld. De prestatie wordt kwantitatief geëvalueerd aan de hand van relevante metrieken zoals Mean Absolute Error (MAE) en Mean Absolute Percentage Error (MAPE).

% Snippet voor een afbeelding dat je bv kan gebruiken voor een Gantt-diagram.
%
% We gebruiken hier de figure*-omgeving zodat de figuur over beide kolommen
% gespreid wordt voor betere leesbaarheid. Probeer de positionering van
% figuren niet te manipuleren (met bv [ht!]), maar zorg altijd voor een
% zinvol bijschrift en label, en refereer er naar in de tekst.
%
% Bij afbeeldingen die je overneemt, sluit je het bijschrift af met een
% bronvermelding (commando \autocite).
%
% \begin{figure*}
    %   \centering
    %   \includegraphics[width=\textwidth]{example-image-16x9}
    %   \caption{\label{fig:gantt}Gantt diagram met de verschillende fasen en milestones van het onderzoek.}
    % \end{figure*}

\section{Verwacht resultaat, conclusie}
\label{sec:verwacht_resultaat_conclusie}

Op basis van de literatuurstudie wordt verwacht dat het computer vision-systeem een haalbaar en kostenefficiënt alternatief vormt voor het commerciële HDTS-systeem. De totale kostprijs van het prototype wordt op minder dan \euro{}500 geschat, wat een aanzienlijke besparing betekent ten opzichte van de huidige standaard.

Wat de modelprestaties betreft, is de doelstelling om een afwijking van minder dan 1 seconde voor de ToF te bereiken. Voor HD streeft het systeem ernaar om in minimaal 80\% van de gevallen de juiste landingszone te classificeren. Figuur~\ref{fig:mockup_confusion} toont een mock-up visualisatie van deze verwachte prestatie voor HD door middel van een verwarringsmatrix.

\begin{figure}
    \centering
    \includegraphics[width=1\linewidth]{"./img/mockup_hd_matrix_11zones.jpg"}
    \caption{Mock-up: Verwarringsmatrix die de verwachte classificatienauwkeurigheid van de HD-zones toont.}
    \label{fig:mockup_confusion}
\end{figure}

De meerwaarde van dit onderzoek ligt in het toegankelijk maken van sporttechnologie. Het voornaamste resultaat is een werkende \emph{Proof of Concept} die aantoont dat een betaalbaar en nauwkeurig alternatief mogelijk is.